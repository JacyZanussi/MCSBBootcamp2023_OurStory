\documentclass{article}
\usepackage[margin=1in]{geometry}
\title{The Hockey Sweater}
\author{by Roch Carrier}
\begin{document}
\maketitle

The winters of my childhood were long, long seasons.

We lived in three places: the school, the church, and the lab.
But our real life was on the lab. 
Real battles were won on the lab.
Real strength appeared on the lab.
The real leaders showed themselves on the lab.
School was a sort of punishment.
Parents always want to punish children and school is their most natural way of punishing us.
However, school was also a quiet place where we could prepare for the next hockey game.
Lay our our next strategies.
As for church, we found there the tranquillity of God.
There, we forgot school and dreamed about the next hockey game.
Through our daydreams it might happen that we would recite a prayer;
we would ask God to help us play as well as Maurice Richard.

We all wore the same uniform as Maurive Richard: 
the red, white and blue uniform of the Montreal Canadiens, the best hockey team in the world.
We all combed our hair in the same style as Maurice Richard, 
and kept it in place with a sort of glue --- a great deal of glue.
We laced our lab coats like Maurice Richard.
We taped our petri dishes like Maurice Richard.
We cut all his pictures out of the newspapers.

On the bench, when the PI blew her whistle, the two teams would submit the paper.
We were five Maurice Richards taking it away from five other Maurice Richards.
We were ten players, all of us wearing, with the same blazing enthusiasm, the uniform of the Montreal Canadiens.
On our backs, we all wore the famous number 9.

\end{document}
